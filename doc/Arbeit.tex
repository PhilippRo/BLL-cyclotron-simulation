\documentclass[14pt, a4paper]{report}

\usepackage{ngerman}

\begin{document}

\title{Besondere Lernleistung \\ Simulation eines Zyklotrons }
\author{Philipp Rosendahl \\ philipp.rosendahl@gmx.de \\ Mallinckrodt-Gymnasium Dortmund}
\date{23.12.2016}
\maketitle

\tableofcontents

\chapter{Vorwort}
\section{Warum sollte man ein Zyklotron simulieren?}
Teilchenbeschleuniger erm"oglichen es Forschern einen tiefen Einblick in die
Materie zu erhalten. Man sollte sich daher mit ihnen besch"aftigen, da sie die
experimentelle Grundlage der Teilchenphysik dastellen. Dennoch sind sie sehr
teuer und als Sch"uler oder Laie ist es nur schwierig möglich, einen
Teilchenbeschleuniger zu benutzen. \\
Man sollte Zyklotrons simulieren, da man Fehler vorhersagen kann, ohne die Ressourcen
f"ur einen Testlauf aufbringen zu m"ussen. Als Nichtforscher kann man, Dank Simulationen, einen Einblick in Teilgebiete aktueller Forschung erhalten und risikofrei experimentieren.
\section{Motivation}
Ich möchte diese besondere Lernleistung erbringen, da ich es interessant finde 
zu abstrahieren, Modelle zu entwickeln und diese am PC umzusetzten. 
Teilchenbeschleuniger sind relativ komplex und genie"sen ein gro"ses Interesse
in der "Offentlichkeit, was sie für eine Simulation attraktiv macht. Au"serdem
k"onnte die Simulation im Rahmen des Physikunterrichtes eingesetzt werden, da dort 
ein preiwertes und risikofreies Experimentieren erwünscht ist.
\section{Danksagung}
Ich möchte mich bei allen bedanken, die mich unterst"utzt und ermutigt haben, eine
besondere Lernleistung zu erbringen: Bei Herrn Dr. Unkelbach und Herrn H"orsken, 
die mich betreut haben, bei Herrn Dr. Wiele, Herrn Broelemann und Herrn Wambach, 
die mich auf die Möglichkeit einer besonderen Lenrleistung hingeweisen und mir im
Rahmen von mathematischen Akademien fachlich Exkurse ermöglicht haben und nicht 
zuletzt bei dem Mallinckrodt-Gymnasium, im Speziellem bei Herrn Weishaupt und Herrn
Freudenreich, das mich ausgebildet hat und mir die Möglichkeit gibt diese Arbeit zu
erbringen. Und zuletzt bei meinen Eltern, die mir ihren Möglichkeiten entsprechend
geholfen haben.

\part{Physikalische Grundlagen}
\chapter{Bedeutung eines Zyklotrons}
\section{Aufbau von Teilchenbeschleunigern}
Man differenziert zwischen drei Arten von Teilchenbeschleunigern \footnotemark
\footnotetext{vergleiche Physikbuch \_ Seite ?}
\begin{enumerate}
\item
Der \textbf{Linearbeschleuniger} besteht aus Beschleunigungskondensatoren, die linear
t oder gradlienig hintereinander angeordnet sind, durch die
Teilchen auf eine hohe Geschwindigkeit gebracht werden. Damit ist jeweils ein hoher
Konfigurationsaufwand verbunden, da viel wie z.B der Abstand zwischen allen
Kondensatorplatten oder gegebenenfalls die Frequenz an den Kondensatorplatten 
eingestellt werden muss.
\item
Das \textbf{Zyklotron} besteht aus 2 Halbkreisen zwischen denen ein Kondensator ist.
Es werden 2 Arten von Feldern benutzt. Eimal die magnetischen, mit denen die 
Teilchen auf einer Kreisbahen\footnote{mit der Lorenzkraft}
gehalten werden, und die elektrischen, 
mit denen die Teilchen w"ahrend des "Ubergangs zwischen den Halbkreisen in dem 
Kondensator beschleunigt werden. Ein Zyklotron hat einen geringen 
Konfigurationsaufwand, da nur 3 nach der klassischen Vorstellung konstante Gr"o"sen
einzustellen sind: Die Masse, die Frequenz und das Magnetfeld an den Hablkreisen. 
\item
Der \textbf{Synchrotron} ist im Prinzip wie ein Zyklotron aufgebaut mit dem 
Unterschied, dass das Magnetfeld anpassbar ist. Experimente haben gezeigt, dass die
Masse nach Einsteins Relativit"atstheorie keine konstante Größe ist. Daher muss 
dann w"ahrend der Beschleunigung das Magnetfeld angepasst werden.
\end{enumerate}

\section{Das Zyklotron im Vergleich zu anderen}
Zyklotrons sind die einfachsten Teichenbeschleuniger. Die anderen 
Teilchenbeschleunigern, doch die sind meistens deutlich schwieriger zu benutzten. 
Ein Nachteil des Zyklotrons ist zum Beispiel, das ein Synchrotron die Teilchen auf 
höhere Geschwindigkeiten beschleunigen kann. Damit eignet sich ein Zyklotron für 
Massenspektrometrien und andere Experimente, bei denen man nur wenig am 
Beschleuniger anpassen möchte und relativ geringe Geschwindigkeiten ausreichen. 
Bei einem Linearbeschleunigern müssen zum Beispiel, anders als beim Zyklotron, von 
Teilchenart zu Teilchenart, mehrere Beschleunigungskondensatoren an ihrem 
Plattenabstand eingestellt werden, wobei man von Kondensator zu Kondensator andere
Abst"ande braucht. Ein Synchrotron braucht eine dauerhafte Anpassung eines 
Magenetfeldes, was ein Zyklotron nicht benötigt

\chapter{Funktion eines Zyklotrons}
\section{Das elektrische Feld}
\section{Das magnetische Feld}
\section{Mit elektrischen und magentischen Feldern Ionen beschleunigen}


\chapter{relativistische Mechanik im Zyklotron}
\section{Einsteins Konstante}
\section{Experimenteller Befund: Das klassische Modell schl"agt fehl}
\section{Neues Modell unter Ber"ucksichtigung der relativistischen Mechanik}

\chapter{verwendetes Modell}
\section{Betrachtung der Teilchen}
\section{Betrachtung der Felder}
\section{Beschleunigung im E-Feld}
Interpretation der Kraft als Ableitung des Impulses
\begin{equation}
\dot{p} = f 
\end{equation}
Aus der Unkelbachschen Formelsammlung l"asst sich folgender Zusammenhang ableiten:
\begin{eqnarray}
\frac{d}{dt} (\frac{m_0 * v}{\sqrt{1 - (\frac{v}{c}})^2}) && = \frac{U * Q}{d} \\
\frac{d}{dt} (\frac{v}{\sqrt{1 - (\frac{v}{c})^2}}) && = \frac{U * Q}{d * m_0} \\
\int_{t_0}^t \frac{d}{dt} (\frac{v}{\sqrt{1 - (\frac{v}{c})^2}}) 
    && = \int_{t_0}^t \frac{U * Q}{d * m_0} \\
\end{eqnarray}
\begin{center}
$ \frac{U * Q}{d * m_0} $ wird als $ a $ interpretiert
\end{center}
\begin{eqnarray}
\frac{v}{\sqrt{1 - (\frac{v}{c})^2}} && = a * t + v_0 \\
\frac{v^2}{1 - (\frac{v}{c})^2} && = (a * t + v_0 )^2 \\
v^2  && = (a * t + v_0)^2 - (\frac{v}{c})^2 * (a * t + v_0)^2\\
v^2 + (\frac{v}{c})^2 * (a * t + v_0)^2 && = (a * t + v_0)^2 \\
v^2 * (1 + \frac{(a * t + v_0)^2}{c^2}) &&  = (a * t + v_0)^2 \\
v^2 && = \frac{(a * t + v_0)^2}{1 + (\frac{a * t + v_0}{c})^2} \\ 
v && = \frac{a * t + v_0}{\sqrt{1 + (\frac{a * t + v_0}{c})^2}}
\end{eqnarray}
\section{klassische Mechanische Zusammenh"ange im Zyklotron}





\part{Umsetzung am PC}
\chapter{Die technische Herausvorderung}
\section{Geschwindigkeit und Rechenaufwand}
\section{Genauigkeitsproblem bei Flie"skommazahlen}

\chapter{Die Wahlt der Werkzeuge}
\section{Sprachen}
\section{Bibliotheken}
\section{Buildsysteme}

\chapter{Design Grundlagen in C++}
\section{Klassen und Strukturen}
\section{Templates}
\section{Operator"uerladung}
\section{Singletonobjekte}

\chapter{Multitasking}
\section{Warum Multithreading ?}
\section{Wer viel gleich macht, macht fehler.}
\section{Synchronisierung}
\section{Kommunikation zwischen Threads}

\chapter{Desing der Simulation}
\section{Application}
\section{Window} 
\section{Graphen und GraphController}
\section{Zyklotron und ZyklotronController}
\section{Starten der Simulation}

\chapter{Umsetzung}
\section{Lösung des Genauigkeitsproblem bei Flie"skommazahlen}
\section{Channel f"ur die Kommunikation zwischen Threads}
\section{Geometrie im Fenster}
\section{Wie wird ein Graph gerendert}
\section{Starten der Simulation}
\section{Stoppen der Simulation}

\appendix
\chapter{Klassendokumentation}
\newpage

\end{document}
